\documentclass{article}
\usepackage[utf8]{inputenc}
\usepackage[linesnumbered,ruled,vlined]{algorithm2e}
\usepackage{amsmath}

\usepackage[nohead, nomarginpar, margin=1in, foot=.25in]{geometry}


%================================

\title{CSCI 163 - Theory of Algorithms HW2}
\author{Tushar Shrivastav}
\date{January 2023}

\begin{document}

\maketitle

\section{Problem 1}
\textbf{Problem: }\\
Find an exact closed-form formula for A(n), the worst-case number of additions performed by the following algorithm on input n. Do this for both parts a
and b. This includes writing the recurrence relations and solving the recurrences using
backward substitution (or other techniques). Do not use the formulas obtained in class
for solving the recurrence. \\

a)
\begin{algorithm}
\caption{Mystery(n)}
\SetKwInOut{Input}{Input}
\Input{Integer n}
\If{n = 1}{
    \KwRet{n + 5}
}
\KwRet{Mystery(n-1) + Mystery(n-1) + n + 5 + 2}
\end{algorithm} \\
b)
\begin{algorithm}
\caption{Mystery(n)}
\SetKwInOut{Input}{Input}
\Input{Integer n}
\If{n = 1}{
    \KwRet{n + 5}
}
\SetKw{KwBy}{by}
\SetKwFunction{FMystery}{Mystery}
\SetKwData{temp}{temp}
\temp = \FMystery{n-1}
\KwRet{\temp + \temp + n + 5 + 2}
\end{algorithm}

\textbf{Solution: } \\

a)
Recurrence Relation: \\
\[A(n=1) = 1\]
\[A(n) = 2A(n-1) + 4\] \\

Solve with Backwards Substitution: \\
\[A(n-1) = 2[2A(n-2)+4] + 4 = 4A(n-2) + 8 + 4\]
\[A(n-2) = 4[2A(n-2) + 4] + 8 + 4 = 8A(n-3) + 16 + 8 + 4\]

Evident Pattern: \\
\[2^iA(n-i) + 4\sum_{j=0}^{i-1} 2^j\]

Simplify: \\
\[2A(n-i) + 4[\frac{2^i - 1 }{2 - 1}]\]
\[2A(n-i) + 4\cdot 2^i - 4\]

Use the initial condition to solve: \\
\[2^{n-1}A(n-n-1) + 2^{n-1} - 4\]

Thus,
\[5\cdot2^{n-1} - 4 \in \Theta(2^N)\]
b) Recurrence Relation: \\
\[A(n=1)=1\]
\[A(n) = A(n-1) + 4\]

Use Backwards Substitution: \\
\[A(n-1) = A(n-2) + 8\]
\[A(n-2) = A(n-3) + 12\]
\[A(n-1) = A(n-4) + 16\] 

Evident Pattern: \\
\[A(n-i) +4i\]

Simplify:
\[A(n-n-1) + 4(n-1)\]
\[1+4n-4\]

Thus,
\[4n-3 \in \Theta(n)\]

\section{Problem 2}
\textbf{Problem:} \\
\begin{algorithm}
\caption{P(n)}
\SetKwInOut{Input}{Input}
\Input{Integer n}
\If{n = 1}{
\KwRet{2 * n * (n-1)}
}
\For{i = 1 to n}{
temp = temp * i
}
\KwRet{temp * P(n/5) * P(n/5) * P(n/5)}
\end{algorithm} \\
\textbf{Solution: } \\
Recurrence Relation:
\[A(1) = 2\]
\[A(n) = 3 + n + 3A(\frac{n}{5})\]
Let's assume $n = 5^k$, then...
\[A(5^0) = 2\]
\[A(5^k) = 3 + 5^k + 3A(5^{k-1})\]
Using the formula for $A(b^k)$ to solve...
\[ 2 \cdot 3^k + \sum_{i=0}^{k-1} 3^i(3+5^{k-i})\]
\[2 \cdot 3^k + \sum_{i=0}^{k-1} 3\cdot3^{i} + 3^i(\frac{5^k}{5^i})\]
\[2 \cdot 3^k + 3\sum_{i=0}^{k-1} 3^i + 5^k\sum_{i=0}^{k-1} (\frac{3}{5})^i\]
\[2 \cdot 3^k + 3[\frac{3^k - 1}{3-1}] + 5^k[\frac{(\frac{3}{5})^k - 1}{\frac{3}{5} - 1}]\]
\[\frac{4\cdot3^k + 3\cdot3^k - 3 - (5\cdot5^k(\frac{3^k}{5^k})-5\cdot5^k)}{2}\]
\[\frac{2\cdot3^k - 3 - [5\cdot3^k - 5\cdot5^k]}{2}\]
\[\frac{2\cdot3^k + 5\cdot5^k - 3}{2}\] \\
Use $n = 5^k \implies k=\log_5 n$ to solve...
\[\frac{ 2\cdot3^{log_5 n}+ 5\cdot5^{log_5 n} - 3}{2}\] \\
Thus (use formula shown to re-write $3^{log_5 n}$),
\[\frac{ 2\cdot n^{log_5 3}+ 5n - 3}{2} \in \Theta(n)\]


\section{Problem 3}
\textbf{Problem:} \\
Use the master theorem to solve the following recurrence asymptotically for
n.
\[T(1) = 3\]
\[T(n) = 6T(\frac{n}{4}) + f(n)\] \\
(a) where $f(n) = n^2 + \log n$ \\
(b) where $f(n) = n^{\log_4 6}$ \\
(c) where $f(n) = 100n + \sqrt{n}$ \\
\textbf{Solution:} \\
We know $a =6$, $b=2$, and $c=3$ from the problem. Thus, we can apply the master theorem for all 3 parts.\\
(a) We know $d=2$ then $b^d = 4^2 = 16$. Since $ 6 > 16$, then $T(n) \in \Theta(n^2)$. \\
(b) We know $d=\log_4 6$ then $b^d = 4^{\log_4 6}$. Since $6=6$, then $T(n) \in \Theta(n^{log_4 6} \log n)$ \\
(c) We know $d=1$, then $b^d = 4^1 = 4$. Since $6 > 4$, then $T(n) \in \Theta(n^{log_4 6})$
\end{document}
